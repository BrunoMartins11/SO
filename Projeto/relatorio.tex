%%%%%%%%%%%%%%%%%%%%%%%%%%%%%%%%%%%%%%%%%
% University Assignment Title Page 
% LaTeX Template
% Version 1.0 (27/12/12)
%
% This template has been downloaded from:
% http://www.LaTeXTemplates.com
%
% Original author:
% WikiBooks (http://en.wikibooks.org/wiki/LaTeX/Title_Creation)
%
% License:
% CC BY-NC-SA 3.0 (http://creativecommons.org/licenses/by-nc-sa/3.0/)
% 
% Instructions for using this template:
% This title page is capable of being compiled as is. This is not useful for 
% including it in another document. To do this, you have two options: 
%
% 1) Copy/paste everything between \begin{document} and \end{document} 
% starting at \begin{titlepage} and paste this into another LaTeX file where you 
% want your title page.
% OR
% 2) Remove everything outside the \begin{titlepage} and \end{titlepage} and 
% move this file to the same directory as the LaTeX file you wish to add it to. 
% Then add \input{./title_page_1.tex} to your LaTeX file where you want your
% title page.
%
%%%%%%%%%%%%%%%%%%%%%%%%%%%%%%%%%%%%%%%%%
%\title{Title page with logo}
%----------------------------------------------------------------------------------------
%	PACKAGES AND OTHER DOCUMENT CONFIGURATIONS
%----------------------------------------------------------------------------------------

\documentclass[12pt]{article}
\usepackage[english]{babel}
\usepackage[utf8]{inputenc}
\usepackage{amsmath}
\usepackage{graphicx}
\usepackage[labelformat=empty]{caption}
\usepackage[T1]{fontenc}

\begin{document}

\begin{titlepage}

\newcommand{\HRule}{\rule{\linewidth}{0.5mm}} % Defines a new command for the horizontal lines, change thickness here

\center % Center everything on the page
 
%----------------------------------------------------------------------------------------
%	HEADING SECTIONS
%----------------------------------------------------------------------------------------

\textbf{\textsc{\LARGE Universidade do Minho}}\\[1.5cm] % Name of your university/college
\textsc{\Large Mestrado Integrado em Engenharia Informática}\\[0.5cm] % Major heading such as course name
\textsc{\large Sistemas Operativos}\\[0.8cm] % Minor heading such as course title

%----------------------------------------------------------------------------------------
%	TITLE SECTION
%----------------------------------------------------------------------------------------

\HRule \\[0.4cm]
{ \huge \bfseries Processamento de Notebooks}\\[0.4cm]
\Large \emph{2 de junho de 2018} % Title of your document
\HRule \\[1.5cm]
 
%----------------------------------------------------------------------------------------
%	AUTHOR SECTION
%----------------------------------------------------------------------------------------

%\begin{minipage}{0.4\textwidth}
%\begin{flushleft} \large
%\emph{Authors:}\\
%Bruno Martins A.....
%Leonardo Neri A.....
%Márcio Sousa  A82400 % Your name
%\end{flushleft}
%\end{minipage}
%~
%\begin{minipage}{0.4\textwidth}
%\begin{flushright} \large
%\emph{Supervisor:} \\
%Dr. James \textsc{Smith} % Supervisor's Name
%\end{flushright}
%\end{minipage}\\[2cm]

% If you don't want a supervisor, uncomment the two lines below and remove the section above
\Large \emph{Membros do Grupo 66}\\
Bruno Martins A80410
Leonardo Neri A80056
Márcio Sousa A82400
%----------------------------------------------------------------------------------------
%	DATE SECTION
%----------------------------------------------------------------------------------------



%----------------------------------------------------------------------------------------
%	LOGO SECTION
%----------------------------------------------------------------------------------------

%\includegraphics{logo.png}\\[1cm] % Include a department/university logo - this will require the graphicx package
 
%----------------------------------------------------------------------------------------

\vfill % Fill the rest of the page with whitespace

\end{titlepage}

\section{Introdução}
Este projeto teve como objetivo criar um programa na linguagem de programação C que permitisse o processamento de \textbf{\textit{notebooks}}. Um \textbf{\textit{notebook}} é um ficheiro de texto que contém comentários e comandos e que depois do seu processamento também contém o resultado dos comandos.
O conteúdo do ficheiro, como referido anteriormente, contém um comando, que começa obrigatoriamente com o símbolo \textit{\$}, e o seu output está delimitado entre > > > output < < < . O \textbf{\textit{notebook}} também poderá ter nalguns casos um número logo após o símbolo \textit{\$} referente ao output do \textit{n} comando anterior que deverá ser um argumento do próximo comando a ser executado. 

\section{Estruturas de Dados}



\subsection{COMMAND}

Para o armazenamento do conteúdo dos \textbf{\textit{notebooks}} foi criada uma estrutura \textit{COMMAND} que contém quatro campos. Uma \textit{String} \textit{\textit{input}} que contém o comando a executar, uma \textit{String} \textit{\textit{output}} que contém o output do comando, uma \textit{String} \textit{\textit{comment}} que contém o comentário que está imediatamente antes de comando, e um \textit{inteiro} que se refere ao número do output anterior que deve ser utilizado como \textit{input} no comando atual. 

\subsection{LIST}

Para assegurar uma coneção entre todas as células \textit{COMMAND} foi utilizada uma lista ligada. Esta lista está ordenada tendo em conta a organização do \textit{\textit{notebook}} a ser processado.


\section{Execução}

\subsection{Leitura e parsing do ficheiro}
A execução do programa começa pela abertura do ficheiro passado como argumento com a \textit{system call} \textit{\textit{open}}, de seguida é lido o texto que lá está dentro, linha a linha, um caracter de cada vez usando o \textit{\textit{read}}. Logo após ser finalizada a leitura da linha, o conteúdo dela é processada por uma função de \textit{parsing} que verifica se é um comando, um comentário, um  output e também o \textit{inteiro} a seguir ao \textit{\$} e insere no campo designado da estrutura \textit{COMMAND}.

\subsection{Execução de Comandos}
No que toca à execução dos comandos propriamente dita, o programa cria dois pipes, para que na criação de um processo filho que executa os comandos, este possa comunicar com o processo pai de forma a redirecionar inputs e outputs. Seguidamente é feito um \textit{\textit{fork}} que cria um processo filho e onde os comandos são executados usando \textit{\textit{execvp}}. Os outputs do comando são redirecionados para um pipe anteriormente criado e guardados na respetiva posição da estrutura. Na eventualidade de um comando necessitar de um output que não seja o imediatamente anterior a ele, é encontrado esse output e escrito no \textit{file descriptor} de onde o processo vai ler os argumentos para execução. Finalmente no fim de todo este processo os resultados são guardados na estrutura. No caso de ser necessário um output de um comando que não seja o imediatamente anterior, é percorrida a estrutura \textit{LIST} até ele ser encontrado e é escrito no pipe de onde o comando vai ler antes de executar.


\subsection{Escrita no ficheiro}
Visto que a estrutura em que está guardado o ficheiro e os outputs dos comandos dentro dele está ordenada segundo a organização do ficheiro, a escrita no ficheiro é feita utilizando a função \textit{\textit{creat}}, criando assim um ficheiro novo igual ao inicialmente passado mas agora sem conteúdo, são percorridos todos os elementos da estrutura \textit{LIST} criada e escritas todas as conponentes dessa estrutura.

\section{Controlo de Erros}

Quanto ao controlo de erros neste programa têm de ser considerados alguns cenários que possam ocorrer. Numa primeira instância pode não ser passado nenhum argumento ao programa bem como mais que um, nesse caso o programa é interrompido. Outro caso possível é, a executar um comando cujo input seja o output de um \textit{N} comando anterior, o número seja maior que o número de comandos já executados logo não pode ser executado esse comando tendo o programa de terminar. Erros de leitura e/ou escrita em ficheiros foram também tidos em conta bem como na alocação de memória para as estruturas. Na criação de pipes é verificado se estes são gerados sem erros e aquando da chamada da função \textit{fork} também ela sofre controlo de erros.

\section{Conclusão}
Este projeto foi uma excelente oportunidade para aplicar conhecimentos relativos a Sistemas Operativos, tendo desta forma ficado mais vincado como e o porquê de um sistema operativo executar processos de forma eficiente. Quanto a melhoramentos possíveis ao programa acima descrito poderia ser melhorado através da execução concorrente de comandos. Esta arquitetura necessitaria no entanto de outro tipo de cuidados tal como a verificação de dependências de execução (um comando para executar necessitar do output de um comando que ainda não foi executado). Contudo, na opinião dos integrantes do grupo, os objetivos do trabalho foram atingidos pois todas as funcionalidades do programa executavam em conformidade com o requerido. 

\end{document}