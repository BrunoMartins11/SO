%%%%%%%%%%%%%%%%%%%%%%%%%%%%%%%%%%%%%%%%%
% University Assignment Title Page 
% LaTeX Template
% Version 1.0 (27/12/12)
%
% This template has been downloaded from:
% http://www.LaTeXTemplates.com
%
% Original author:
% WikiBooks (http://en.wikibooks.org/wiki/LaTeX/Title_Creation)
%
% License:
% CC BY-NC-SA 3.0 (http://creativecommons.org/licenses/by-nc-sa/3.0/)
% 
% Instructions for using this template:
% This title page is capable of being compiled as is. This is not useful for 
% including it in another document. To do this, you have two options: 
%
% 1) Copy/paste everything between \begin{document} and \end{document} 
% starting at \begin{titlepage} and paste this into another LaTeX file where you 
% want your title page.
% OR
% 2) Remove everything outside the \begin{titlepage} and \end{titlepage} and 
% move this file to the same directory as the LaTeX file you wish to add it to. 
% Then add \input{./title_page_1.tex} to your LaTeX file where you want your
% title page.
%
%%%%%%%%%%%%%%%%%%%%%%%%%%%%%%%%%%%%%%%%%
%\title{Title page with logo}
%----------------------------------------------------------------------------------------
%	PACKAGES AND OTHER DOCUMENT CONFIGURATIONS
%----------------------------------------------------------------------------------------

\documentclass[12pt]{article}
\usepackage[english]{babel}
\usepackage[utf8]{inputenc}
\usepackage{amsmath}
\usepackage{graphicx}
\usepackage[labelformat=empty]{caption}

\begin{document}

\begin{titlepage}

\newcommand{\HRule}{\rule{\linewidth}{0.5mm}} % Defines a new command for the horizontal lines, change thickness here

\center % Center everything on the page
 
%----------------------------------------------------------------------------------------
%	HEADING SECTIONS
%----------------------------------------------------------------------------------------

\textbf{\textsc{\LARGE Universidade do Minho}}\\[1.5cm] % Name of your university/college
\textsc{\Large Mestrado Integrado em Engenharia Informática}\\[0.5cm] % Major heading such as course name
\textsc{\large Programação orientada a objetos}\\[0.5cm] % Minor heading such as course title

%----------------------------------------------------------------------------------------
%	TITLE SECTION
%----------------------------------------------------------------------------------------

\HRule \\[0.4cm]
{ \huge \bfseries Java Fatura}\\[0.4cm] % Title of your document
\HRule \\[1.5cm]
 
%----------------------------------------------------------------------------------------
%	AUTHOR SECTION
%----------------------------------------------------------------------------------------

%\begin{minipage}{0.4\textwidth}
%\begin{flushleft} \large
%\emph{Authors:}\\
%Bruno Martins A.....
%Leonardo Neri A.....
%Márcio Sousa  A82400 % Your name
%\end{flushleft}
%\end{minipage}
%~
%\begin{minipage}{0.4\textwidth}
%\begin{flushright} \large
%\emph{Supervisor:} \\
%Dr. James \textsc{Smith} % Supervisor's Name
%\end{flushright}
%\end{minipage}\\[2cm]

% If you don't want a supervisor, uncomment the two lines below and remove the section above
\Large \emph{Membros do Grupo 66}\\
Bruno Martins A80410
Leonardo Neri A80056
Márcio Sousa A82400
%----------------------------------------------------------------------------------------
%	DATE SECTION
%----------------------------------------------------------------------------------------



%----------------------------------------------------------------------------------------
%	LOGO SECTION
%----------------------------------------------------------------------------------------

%\includegraphics{logo.png}\\[1cm] % Include a department/university logo - this will require the graphicx package
 
%----------------------------------------------------------------------------------------

\vfill % Fill the rest of the page with whitespace

\end{titlepage}

\section{Introdução}
Este projeto teve como objetivo criar um progrma em na linguagem de programação C que permitisse o processamento de \textbf{\textit{notebooks}}. Um \textbf{\textit{notebook}} é um ficheiro de texto que contém comentarios e comandos e que depois do seu processamento também contém o resultado dos comandos.
O conteúdo do ficheiro, como referido anteriormente, contém um comando, que começa obrigatoriamente com o símbolo \textit{\$}, e o seu output está delimitado entre \textit{>>> <<<}.  

\section{Estruturas de Dados}



\subsection{COMMAND}

Para o armazenamento do conteúdo dos \textbf{\textit{notebooks}} foi criada uma estrutura \textit{COMMAND} que contém três campos. Uma \textit{String} \textit{\textit{input}} que contém o comando a executar, uma \textit{String} \textit{\textit{output}} que contém o output do comando e uma \textit{String} \textit{\textit{comment}} que contém o comentário que está imediatamente antes de comando. 

\subsection{LIST}

Para assegurar uma coneção entre todas as células \textit{COMMAND} foi utilizada uma lista ligada. Esta lista está ordenada tendo em conta a organização do \textit{\textit{notebook}} a ser processado.


\section{Execução}

\subsection{Leitura e parsing do ficheiro}
A execução do programa começa pela abertura do ficheiro passado como argumento com a \textit{system call} \textit{\textit{open}}, de seguida é lido o texto que lá está dentro linha a linha, um caracter de cada vez usando o \textit{\textit{read}}. Logo após ser finalizada a leitura da linha, o conteúdo dela é processada por uma função de \textit{parsing} que verifica se é um comando, um comentário ou um output e insere no campo designado da estrutura \textit{COMMAND}.

\subsection{Empresa}
\begin{itemize}
\item \textbf{\underline{Obter as faturas da empresa ordenadas por data ou valor:}} o procedimento é o mesmo que o que permite aos contribuintes verem todas as faturas emitidas em seu nome, mas desta vez, aplicado a uma empresa. Depois de ter a lista das faturas, esta é ordenada pela ordem desejada.
\item \textbf{\underline{Obter a lista de faturas por contribuinte num intervalo de datas:}} as faturas da empresa em questão são filtradas de acordo com o NIF cliente, e depois por data, sendo retornadas a lista de faturas que corresponde a esses dois critérios.
\item \textbf{\underline{Obter a lista de faturas por contribuinte ordenadas por valor de despesa:}} as faturas da empresa em questão sao filtradas por NIF do contribuinte, e é retornada a lista de Faturas ordenada por valor decrescente de despesa, recorrendo a um comparator.
\item \textbf{\underline{Total faturado por uma empresa:}} é percorrida a lista das faturas da empresa, somando o valor das faturas que estão dentro do perído de tempo estipulado, sendo retornado o valor total.
\end{itemize}

\subsection{Admin}
\begin{itemize}
\item \textbf{\underline{Relação 10 contribuintes que mais gastam:}} é percorrido o HashMap que contém todas as identidades fiscais e são filtrados todos os contribuintes individuais para uma lista que é ordenada segundo os gastos de cada um deles. No final são mantidos apenas na lista os primeiros dez contribuintes.

\item \textbf{\underline{Relação 10 empresas que mais faturam:}} é percorrido o HashMap que contém todas as identidades fiscais e são filtradas todas as Empresas para uma lista que é ordenada segundo o total faturado de cada um deles. No final mantemos apenas na lista os primeiros dez contribuintes
\end{itemize}



\section{Conclusão}
Este projeto foi uma excelente oportunidade para aplicar conhecimentos relativos ao paradigma de programação orientada a objetos, usando Java como linguagem de desenvolvimento do projeto. Relativamente a possíveis aspetos a melhorar no projeto, salienta-se que poderia ter sido usado um algoritmo de cálculo de dedução mais específico, e uma lista de concelhos que abrangesse todos os concelhos do país.

\end{document}